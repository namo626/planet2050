@techreport{hamilton2018alternatives,
  title={Alternatives to Manage Sediment at the Intersection of the Gulf Intracoastal Waterway (GIWW) and the Corpus Christi Ship Channel (CCSC)},
  author={Hamilton, Paul and Wood, Eric and Lin, Lihwa and Campbell, Tricia and Olson, Leslie and Jones, Seth and Howard, Steven and Skalbeck, Kathy},
  year={2018},
  institution={ENGINEER RESEARCH AND DEVELOPMENT CENTER}
}

@article{brown2004simulating,
  title={Simulating larval supply to estuarine nursery areas: how important are physical processes to the supply of larvae to the Aransas Pass Inlet?},
  author={Brown, CA and Holt, SA and Jackson, GA and Brooks, DA and Holt, GJ},
  journal={Fisheries Oceanography},
  volume={13},
  number={3},
  pages={181--196},
  year={2004},
  publisher={Wiley Online Library}
}

@article{jenkins1997temporal,
  title={Temporal and spatial variability in recruitment of a temperate, seagrass-associated fish is largely determined by physical processes in the pre-and post-settlement phases},
  author={Jenkins, Gregory P and Black, Kerry P and Wheatley, Melissa J and Hatton, David N},
  journal={Marine Ecology Progress Series},
  volume={148},
  pages={23--35},
  year={1997}
}

%abstract
\noindent
\fcolorbox{black}{azulcielo}{
  \begin{minipage}[H]{.96\linewidth}
    \begin{center}
      \vspace{1cm}
      \section*{Abstract}
We present a study of the potential impact of deepening the Corpus Christi Ship Channel through Aransas Pass; in particular, we study the effect on the transport of red drum fish larvae due to the change in channel depth. The study uses TACC's Stampede2 in order to run a high resolution ocean circulation model called ADCIRC to simulate the seawater entering and exiting the pass for the current and proposed Ship Channel depths. ADCIRC's parallelized structure along with Stampede2's Intel Skylake nodes make it possible to model a domain large enough to account for the effects of incoming winds and tides: the entire Gulf of Mexico and the North American Atlantic coast, while also providing enough resolution to account for changes in the Ship Channel depth: resolution of up to 10 m.  The corresponding transport of larvae modeled as passive particles due to the sea water circulation is established by releasing particles in the nearshore region outside Aransas Pass and subsequently tracking their trajectories. We compare the difference in the number of larvae that successfully reach appropriate nursery grounds inside Aransas Pass for four distinctive initial larvae positions in 
the nearshore region. Our results indicate that the change in channel depth does not significantly alter the number of red drum larvae that reach suitable nursery grounds, overall, across all considered scenarios, we see a net increase of 0.5\%.  
      \vspace{1cm}
    \end{center}
  \end{minipage}
  
}